% Abstract for the TUM report document
% Included by MAIN.TEX


\clearemptydoublepage
\phantomsection
\addcontentsline{toc}{chapter}{Abstract}	





\vspace*{2cm}
\begin{center}
{\Large \bf Abstract}
\end{center}
\vspace{1cm}

\paragraph*{} This bachelor thesis presents a new approach for Feature and Label Selection for Multi-Label Classification. In general, classification is the task of assigning one or more labels to an unseen instance. Multi-label classification is concerned with the assignment of a set of labels is assigned to each instance. This is useful for applications such as protein function annotation and prediction of toxicological endpoints. This work bases on the assumption that there exist subsets of the label and feature space having a higher internal dependence than to the rest of data. Here, two methods are proposed to identify those groups and apply any multi-label learning scheme on it. It can be shown that splitting the dataset into subgroups does not affect the performance, yet it does yield improvements in some cases.

\paragraph*{} Diese Bachelorarbeit zeigt einen neuen Ansatz zur Feature und Label-Auswahl zur Multi-Label Klassifizierung. Allgemein beschreibt Klassifizierung die Aufgabe unbekannten Objekten eine oder mehrere Klassen zuzuordnen. Dieser Schritt wird Multi-Label Klassifizierung genannt, wenn jedem Objekt eine Menge von Klassen zugeordnet wird. Dies ist unter anderem n\"utzlich zur Protein-Funktionsanalyse sowie zur Bestimmung der Toxizit\"at unklassifizierter Substanzen. Diese Arbeit basiert auf der Annahme, dass innerhalb dieser Daten Untergruppen von Features und Labels bestehen die einen gr\"o\ss{}eren Zusammenhang aufweisen als zu den restlichen Attributen. Zwei Methoden wurden entwickelt um diese Untergruppen zu identifizieren und Multi-Label Klassifikatoren auf diesen zu lernen. Es konnte gezeigt werden, dass die Zerteilung der Feature- und Label-Menge keinen negativen Einfluss auf die Vorhersagegenauigkeit hat. In einigen F\"allen konnte sogar eine Verbesserung festgestellt werden. 