\section{Background}

% But what you can already write is a background chapter (2 to 3 pages).
% Points we want to have included are:
%   - Cancer (on the molecular level)
%   - Microarrays
%   - GO terms
%   - GSEA
%   - Fisher's exact test and statistical testing in general

  \subsection[Cancer]{Cancer - on a molecular level}
  \label{sec:cancer}
  
  %tailor this more towards specific cancers, eg. melanoma?
  
  The term cancer describes a broad spectrum of more than 200 diseases with the
common characteristic of uncontrolled cellular growth that may lead to tumor
formation, invasion of surrounding tissues and metastasis. Cancer can be caused
by individual chemical, physical and biological agents and any combinations
thereof with only a few strong causative correlations scientifically proven for
mutagens and carcinogens like e.g. highly energetic ionizing or non-ionizing
radiation, tobacco smoke ingredients, alcohol, aerosolized asbestos, benzene
vapor or aflatoxin B1 but also for physical inactivity, obesity, high-salt diet
as well as viral infections by so-called oncoviruses like the human
papillomavirus and Epstein-Barr virus. One or many of these causative agents,
sometimes in combination with additional hereditary predispositions, can affect
the expression of both oncogenes and tumor suppressor genes. Oncogene products
generally promote cell proliferation and growth and include regulatory GTPases
like Ras and transcription factors like myc, while tumor suppressor genes
(``anti-oncogenes'') generally control apoptosis and inhibit cell division, like
the \textit{TP53} transcript p53 which tightly controls the cell cycle and can
initiate DNA repair as well as apoptosis in case of severe damage to the DNA. 
  
  Possible genetic transformations or mutations range from point mutations like
single nucleotide changes, deletions or insertions (leading to premature
stopping, reading frame extension or shift during translation), allele or copy
number changes by duplication, inversion, deletion or chimerization of entire
protein coding sequences or non-coding control regions as  well as abnormal
splice variants all the way to chromosomal translocations. The resulting
differences both in protein expression levels -- eg. an increased level of
oncogene expression or a decreased level of tumor suppressor gene expression --
and in protein activity across several interconnected molecular pathways can
set of a chain reaction of self-amplifying, compounding errors that influence
important cellular functions such as DNA damage repair, cell proliferation,
apoptosis signaling, inter-cellular signaling (eg. hormone reception and
secretion or antigen presentation) and adhesion to surrounding cells and the
extracellular matrix.
  
  \subsection{DNA Microarrays}
  \label{sec:microarrays}
  
  % saving me the hassle of a disambiguation between all the different types of microarrays by putting a ``DNA'' in front of it
  
  Changes in expression levels of genes between different samples or tissues can
be detected in parallel by immobilizing thousands of microscopic spots of DNA
with known sequences and positions (known as probes, reporters or oligos) to a
glass or silicon surface (known as DNA chip, biochip or DNA microarray).
Multiple identical chips (or -- depending on the technology and amount of
samples -- the very same chip) are hybridized with different samples of
fluoroscope-labeled complementary DNA (cDNA) targets generated by whole-RNA
extraction, reverse transcription, labeling and fragmentation. A washing step
then removes all partially complementary strands, leaving only complementary
probe-target pairs on the chip. These can then be detected and quantified by the
means of laser excitation and fluorescence scanning.
  % include initial data analysis steps: normalisation, internal controls, technical replicates, RMA, ...
  
  The Affymetrix Human Exon 1.0 ST Array in the transcript (gene) version (see
\url{http://www.ncbi.nlm.nih.gov/geo/query/acc.cgi?acc=GPL5175} for details)
allows to distinguish between different isoforms of genes as it includes about
four probes per exon and 40 probes for gene that are collectively called a
probeset. This makes it a useful tool when investigating chimeric genes as those
discovered in human melanoma cultures by Berger et al. \cite{Berger2010}.
  %While the entire array contains 1.4 million
%   probe sets we worked with a dataset reduced to 17,381 probe sets based on the
%   filtering and analysis done by Berger et al. prior to their upload to the
%   repository.

% not so sure about those numbers anymore, might just be the marketing dept. confusing probes with probesets.
  
  
  \subsection{Gene Ontology}
  \label{sec:go}
  
  The Gene Ontology initiative (GO, \url{http://www.geneontology.org/}), a part
of the Open Biological and Biomedical Ontology Foundry (OBO,
\url{http://www.obofoundry.org/}), attempts to unify the representation of genes
and gene product attributes by the means of so-called GO terms which provide a
controlled, species-neutral vocabulary that can be applied to both prokaryotic
and eukaryotic organisms. The terms, each consisting of a unique numeric 7-digit
accession with the prefix ``GO:'' (eg. GO:0008150), a name, a definition and
various fields for cross-references and synonyms, fall into one of three domains
or namespaces: cellular component, molecular function and biological process.
The entire ontology of GO terms is structured as directed acyclic graph (DAG)
with defined relationships between GO terms within a namespace.
  
  A selection of Python scripts called goatools
(\url{https://github.com/tanghaibao/goatools/}) can be used for enrichment of GO
terms. We are using the class \lstinline|GODag| and specifically its method
\lstinline|query_terms| to query the GO term descriptions for the GO IDs
translated by the BridgeDB client.
  
  \subsection{GSEA}
  \label{sec:gsea}
  
  Gene Set Enrichment Analysis (GSEA), a computational method developed by
\cite{Mootha2003} and implemented in a freely available Java software package by
\cite{Subramanian2005} (available at
\url{http://www.broadinstitute.org/gsea/index.jsp}), determines if a defined set
of genes shows statistically significant expression level differences between
two samples. Instead of focussing only on differences in expression levels of
individual genes, their regulatory or functional relationships with other genes
within a gene set are considered as well to allow detection of changes in entire
metabolic pathways, transcriptional programs and stress responses which would be
too subtle and distributed for other statistical tests. The method consists of
the following steps:
  \begin{enumerate}
   \item calculation of enrichment scores (ES)
   
   Scoring gene sets by iterating over a ranked list of all genes and either
increasing a running-sum statistic when a gene is found in the set or decreasing
it when not, with the amount of the increment depending on the correlation of
the gene with the phenotype.
   
   \item estimation of significance level
   
   Instead of permuting genes, the phenotype (or in our case: tissue) labels are
randomly permuted while maintaining the gene correlations and enrichment scores
are recalculated for all permutations to generate an ES null distribution, which
is then used to calculate empirical, nominal \textit{P} values for the observed
ES.
   
   \item adjustments for Multiple Hypothesis Testing
   
   If testing against an entire database of gene sets, the ES values for all
gene sets are normalized (NES) and false discovery rates (FDR) are calculated by
comparing the tails the observed NES distribution and the null NES distribution
calculated from the ES values in step 2.
  \end{enumerate}
  
  The core genes within a gene set that account for most of the enrichment
signal can be identified by analysis of so-called leading-edge graphs which show
the leading-edge subset of genes in the ranked list at or before the maximum
deviation of the running-sum statistic from zero. These biologically significant
subsets of genes can be of special interest when examining curated gene sets
consisting of multiple pathways.
  
  \subsection{Z-Scores}  
  \label{sec:Z-Scores}
  
  The standard score, or so-called z-score, describes how many standard deviations an data point is above or below the mean. Assuming a normal distribution (mean = median = 2. quantile) the z-score therefore indicates how large the distance of a data point from the population mean is.
  With the z-score it is possible to make standardized statements about a bunch of different data sets, like how many data points are equal, above or below the mean. Doing so, one can compare data without referring to the absolute values of the data.

  \subsection{p-Value}
  \label{sec:p-Value}
  
  The p-value indicates the probability of obtaining a certain data point at
least as extreme as actually observed assuming a null hypotheses (e.g. common
normal distribution). By applying a threshold (0.05) one rejects the null
hypotheses if the value lies below the threshold, indicating that such an event
is very unlikely to happen under this distribution.
  
  \subsection{Fisher's exact test}
  \label{sec:fisher}
  
  In order to assign significance to the up regulation we used Fisher's exact test \cite{Fisher1922}. The test is based upon an contingency table counting the number of significant up regulated genes (according to their p-value with an threshold of $0.05$) for each tissue. As Fisher's test uses categorical data (nominal) we labeled up regulated (down regulated) genes, holding an significant low p-value, with "TRUE" where not significant varied genes are labeled "FALSE". Basically Fisher's test computes a probability under the assumption of a hyper geometric distribution, for this exact contingency table under the null hypotheses that all samples are equally distributed. Low probability values therefore indicate that the underlying data set can hardly be explained by one common distribution, hence its likely that there are profound differences.

  Prerequisite for Fisher's test is a large enough sample size (more the 100 samples) for each cell in the contingency table which is given for our data set (about 600 data points).
  
% Everything below is optional, with the only exception of GO. Leaving it as is for now to preserve the text flow and chronology.
   
%   \subsection{GEO SOFT format}
%   \label{sec:geosoft}
%   
%   \lstset{basicstyle=\ttfamily,breaklines=true,keepspaces=true}
%   
%   The \underline{S}imple \underline{O}mnibus \underline{F}ormat in
%   \underline{T}ext (SOFT) is line-based plain text format designed for
%   MIAME-compliant rapid batch submission (and download) of data to the NCBI Gene
%   Expression Omnibus repository. File entries belong to one of the entity types
%   platform, sample or series, each initiated by an entity indicator line or
%   ``caret line'' like \lstinline|^PLATFORM = ...|, \lstinline|^SAMPLE = ...| or
%   \lstinline|^SERIES = ...| (see
%   \url{http://www.ncbi.nlm.nih.gov/geo/info/soft2.html#SOFTformat} for details).
%   
%   A platform section of a SOFT file with Affymetrix data contains - among other
%   entries like platform ID and machine-specific meta-data - a tab-delimited
%   platform table which lists all unique, but file-specific SOFT IDs, their
%   corresponding probe set numbers and a list of Ensembl gene IDs for all human
%   transcripts that are expected to hybridize to the probe set.
%   
%   A sample section consists of subsections with experimental data for each
%   platform that the specific sample was used on. In the case of this practical
%   course, the bang line \lstinline|!Sample_platform_id = GPL5175| identifies all
%   following entries until the next such line as related to the Affymetrix Human
%   Exon 1.0 ST Array described above. Within this subsection the beginning and
%   end of a tab-delimited table with one row per probe set (identified by SOFT
%   ID) and its expression value are marked by the bang lines
%   \lstinline|!sample_table_begin| and \lstinline|!sample_table_end|.
%   
%   While a rudimentary parser of SOFT formatted files for Python is available in
%   the Biopython package (see
%   \url{http://biopython.org/DIST/docs/tutorial/Tutorial.html#htoc124}), the lack
%   of good documentation for both the file format and the parser has driven our
%   decision to write our own simple parser from scratch.
%   
%   \subsection{Ensembl}
%   \label{sec:ensembl}
%   
%   Ensembl (\url{http://www.ensembl.org/}) is a joint project between the
%   European Bioinformatics Institute (EMBL-EBI, \url{http://www.ebi.ac.uk/}) and the
%   Wellcome Trust Sanger Institute (\url{http://www.sanger.ac.uk/}) to develop a
%   software system which produces and maintains a Perl-based automatic annotation
%   on currently 71 eukaryotic genomes enhanced by manual, case-by-case curation
%   of transcripts in important species such as human (current assembly: GRCh37),
%   mouse (GRCm38) and zebrafish (Zv9). All annotated sequences are identified by
%   numeric 11-digit Ensemble gene IDs with alphabetic 4-digit prefixes that
%   distinguish between genes (ENSG), spliced transcripts (ENST), exons (ENSE) and
%   proteins (ENSP).
%   
%   The probe set target transcripts of the Affymetrix microarray used in this
%   course are identified by lists of Ensembl gene IDs with ENSG prefix that are
%   later translated into the corresponding gene IDs for the databases Entrez Gene, 
%   UniProt and Gene Ontology (GO) using a BrigdeDB client.
%   
%   \subsection{Entrez}
%   \label{sec:entrez}
%   
%   Entrez Gene (\url{http://www.ncbi.nlm.nih.gov/gene/}) is database for
%   gene-specific information at the National Center for Biotechnology Information
%   (NCBI, \url{www.ncbi.nlm.nih.gov/}). It aggregates information from other NCBI
%   databases such as RefSeq and GenBank and external sources such as Gene
%   Ontology (see below) and the International Nucleotide Sequence Database
%   Collaboration (INSDC, \url{http://www.insdc.org/}) into gene-centered reports. It is one of many
%   federated databases in the NCBI's Entrez meta search engine.
%   
%   The GeneID, a 1- to 9-digit integer with optional prefix ``GeneID:'' is used
%   to identify individual entries and aggregate information across the NCBI
%   databases. GeneIDs are stable, unique and species-specific, so genes for
%   homologous genes in two closely related organisms will have different numbers
%   (\cite{Maglott2011}). Two number generators with separate integer ranges - one
%   from 7,000,000 to 99,999,999 and another one for numbers \textgreater
%   100,000,000 - are currently used to create new GeneIDs so gaps in batch
%   submissions are expected.
%   
%   \subsection{UniProt}
%   \label{sec:uniprot}
%   
%   UniProt is the \underline{uni}versal \underline{prot}ein sequence and
%   annotation repository maintained by the UniProt Consortium, which currently
%   consists of the European Bioinformatics Institute (EMBL-EBI,
%   \url{http://www.ebi.ac.uk/}), the Swiss Institute of Bioinformatics (SIB,
%   \url{http://www.isb-sib.ch/}) and the Protein Information Resource at
%   Georgetown University (PIR, \url{http://pir.georgetown.edu/}). It combines the
%   manually annotated and reviewed
%   Swiss-Prot database (539,829 sequence entries) and the automatically annotated,
%   unreviewed TrEMBL database (33,106,277 sequence entries) into the UniProtKB.
%   Further databases include UniRef, a repository of sequence clusters for
%   sequence similarity searches, and UniParc, a sequence archive keeping track of
%   identifiers.
%   
%   The Alphanumeric 6-digit UniProt accession numbers (see
%   \url{http://www.uniprot.org/manual/accession_numbers} for details) listed in
%   the ``Entry Information'' section of each database entry are stable from
%   inclusion in UniProt to deletion from UniProt and remain
%   listed even when entries are merged, split or promoted from TrEMBL to
%   Swiss-Prot. However, the first entry in the
%   list, called primary or citable accession number, should be preferred over the
%   secondary accession numbers when citing or linking to entries. Accession
%   numbers are always listed in alphanumerical order.
  

