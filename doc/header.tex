\documentclass[
  a4paper,
  10pt,
  oneside,			% or twoside
  final,			% or draft
%   draft,			% or final
%   titlepage,
  onecolumn,
  openbib,
  headsepline			% Trennlinie im Seitenkopf
  %footsepline,			% Trennlinie im Seitenfuß
%   liststotoc,			% Verzeichnisse im Inhaltsverzeichnis		[obsolete]
%   bibtotoc			% Literaturangaben im Inhaltsverzeichnis	[obsolete]
]{scrartcl}

% Page Layout
\usepackage[usenames,svgnames]{xcolor}

\usepackage[top=3cm,bottom=3cm]{geometry}	% make better use of the page than plain KOMA-script
% left=2cm,right=1.5cm
\usepackage{pdflscape}          % landscape-Format mit PDF
\usepackage{fancyhdr}		% für schicke(?) Seitentitel
\usepackage[pdftex]{graphicx}
\usepackage{pdfpages}		% inclusion of PDFs
\usepackage{enumerate}		% Aufzaehlungen mit alphabetischen oder roemischen Zeichen
\usepackage{setspace}		%provides double spacing or onehalf spacing


% Fonts and Symbols

\usepackage{textcomp}		% Sonderzeichen wie Trademark und Co., aufrechtes \textmu durch \textmu
% \usepackage{wasysym}		% Sonderzeichen wie z.B. \female und \male, \diameter !! Konflikt mit AMS Symbolen!!

% Language

\usepackage{ucs}
\usepackage[utf8x]{inputenc}
\usepackage[english]{babel}	% least to most frequent language
\usepackage{fontenc}

% Sourcecode

\usepackage{upquote}		% correct apostrophe use in code environments
\usepackage{fancyvrb}
\usepackage{listings}
\lstset{language=bash,
  numbers=left,
  numberstyle=\tiny,
  numbersep=5pt,
  breaklines=true,
  frame=l,
%   frame=shadowbox,
%   rulesepcolor=\color{Gray},
  backgroundcolor=\color{White},
  keywordstyle=\color{Blue},
  commentstyle=\color{LightSlateGray},
  stringstyle=\color{Red},
  basicstyle=\small\color{Black},
  tabsize=4,
  showstringspaces=false
}

% Colors

% \usepackage{color}                      %Text in Farben
% \usepackage{xcolor}


% Science


\usepackage{amsthm}
\usepackage{amsmath}		% Konflikt mit Paket ??? bei Integralsymbolen
\usepackage{amssymb}		% AMS packages for math symbols
\usepackage[tight]{units}	%provides correct spacing of units and nice fractions, tight/loose
\usepackage{texshade}		%alignments (ALN, FASTA, ...)


% Tables

\usepackage{array}		%weitere Optionen für Tabellen
\usepackage{multirow}		%Zellen über mehrere Reihen mit \multirow
\usepackage{longtable}		%lange Tabellen, angeblich wesentlich besser als supertabular
%\usepackage{supertabular}	%Umbrechen langer Tabellen in Einzeltabellen
\usepackage{rotating}		%Tabellen und andere floats drehen
\usepackage{ctable}		%midrule

% Lists

\usepackage{mdwlist}		% erlaubt mit itemize* auch Listen ohne Zeilenabstand

% Links and References

\usepackage[english]{varioref}		% advanced references using \vref

% Einbindung von SVG Graphiken via Inkscape ab Version 0.48.1-2

\newcommand{\executeiffilenewer}[3]{%
  \ifnum\pdfstrcmp{\pdffilemoddate{#1}}%
  {\pdffilemoddate{#2}}>0%
  {\immediate\write18{#3}}\fi%
}
% set inkscape binary path according to operating-system
\IfFileExists{/dev/null}{%
  \newcommand{\Inkscape}{/usr/bin/inkscape}%
  }{%
  \newcommand{\Inkscape}{"C:/Program Files (x86)/Inkscape/inkscape.exe"}%
}
% includesvg[scale]{file} command
\newcommand{\includesvg}[2][1]{%
  \executeiffilenewer{#1.svg}{#1.pdf}{%
  \Inkscape -z -D --file="#1.svg" --export-pdf="#1.pdf" --export-latex}%
  \scalebox{#1}{\input{#2.pdf_tex}}%
}

% LETZTES USEPACKAGE IM HEADER!
\usepackage[
  pdftex,
  breaklinks=true,
  colorlinks=true,		% surrounds the links by color frames (false) or colors the text of the links (true)
  citecolor=black,		% color of citation links
  urlcolor=blue,		% color of external links
  linkcolor=blue,		% color of internal links
  filecolor=blue,		% color of file links
%   hidelinks,			% hide link color and border
  plainpages=false
 ]{hyperref}
\urlstyle{same}			% comment out for monospaced link text
% LETZTES USEPACKAGE IM HEADER!