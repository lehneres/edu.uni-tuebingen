\section{Discussion}
As seen already in the heat map (figure \ref{fig:heatmap}), the tissue
expression values were quite heterogeneous especially the ones of samples
GSM433783 and GSM433779. The Fischer's exact test provides additional evidence
that GSM433779 has a unique pattern of gene expression, but a closer examination
of the affected KEGG pathways shows that these differences are all related to
non-carcinogenic pathways. As a consequence of these findings tissue GSM433779
has been excluded from further pathway analysis.

In contrast, a closer examination of differentially regulated pathways in sample
GSM433783 yields several pathways containing oncogenes. ``Pathways in cancer''
and ``Small cell lung cancer'' are obvious reasons candidates, as is the ``Focal
adhesion'' pathway due to the effect of intercellular contacts and adhesion to
the extracellular matrix (ECM)) on the ability of cancerogenic cells to enter
metastasis. The loss of self induced cell death upon losing connection to
neighboring cells and the resulting motility of cancerogenic cells are important
steps in oncogenesis. Contrary to this the last of the examined pathways --
``Lysosome'' -- to our knowledge has no major impact on cancer proliferation.

The three cancer related pathways ``Pathways in cancer'', ``Small cell lung
cancer'' and ``Focal adhesion'' were confirmed to be significantly deregulated
by the GSEA method. But GSEA also provides additional information by exhibiting
that these pathways are under-expressed in GSM433783 compared to all other
tissues. To identify genes with the strongest contribution, pathway
visualizations were created using BiNA which were manually inspected for
interesting candidates (see figures \ref{fig:pathwaysincancer}, \ref{fig:SCLC}
and \ref{fig:focAd}). The target genes \textit{myc}, \textit{cdk2} and
\textit{catenin} can be identified as significantly under-expressed compared to
the average of all remaining tissues.

The protein \textit{Cdk2} is of major importance to promote the cell cycle and
can be an oncogen if over expressed\cite{Du2004}. High expression levels would
result in enhanced proliferation. \textit{Myc} is also a known
oncogene\cite{Dang2005} with a central role in oncogenesis. It regulates the
expression of several cell cycle promoting proteins, one example being shown in
the BiNA pathway of ``Pathways in cancer'' (figure \ref{fig:pathwaysincancer}),
where \textit{myc} promotes \textit{cdk2}. Over-expression of \textit{myc} is a
characteristic feature of tumor cells leading to over-proliferation. Within the
``Focal adhesion'' (figure \ref{fig:focAd}) pathway \textit{catenin} was found
as being down-regulated. Over-expression of \textit{catenin} is strongly
associated with cancer\cite{MacDonald2009} due to the genes' nuclear
localization and possible interaction with multiple transcription factors
inducing proliferation.

We conclude that due to the significantly lower expression values for several
oncogenes and related pathways cancer cells of tissue GSM433783 may have a lower
proliferation rate than the other eight cell lines. Although the aforementioned
genes play major roles in cancer, their lower expression may not necessarily
have a very strong effect on the proliferation of these specific cancer cells.
Enhanced proliferation is only one of many aspects of cancer and normally many
alternative pathways exist in the cellular regulatory network to compensate
down- and up-regulation in individual genes or pathways. Other important
properties of cancer cells such as independence of correct cell contacts,
motility and immunity to apoptosis may not be affected at all.


\subsection{Biological significance}
The given dataset contained nine melanoma tissues but no healthy control tissue.
Thus, all analysis steps implemented in the course could only be applied to
compare expression levels and regulation of pathways between the nine different
cancer tissues, but not with a healthy control tissue. Such a comparison is
likely to reveal that the overall difference between the nine cell lines is of
minor magnitude. The outcome of our analysis therefore has limited biological
relevance and the differences in pathway regulation are expected to be mostly
insignificant. % not a nice word to end it, but i guess it fit's quite well ;-)

% \subsection{possible improvements}
% 
% asTable() output should have tissue-specific expression values in first
% columns rather than last to allow addition of further gene IDs to the DataSet
% without redefining data columns for R heatmap script; alternatively wrap call of
% Rscript in a method that detects the table layout and sets parameters
% accordingly.